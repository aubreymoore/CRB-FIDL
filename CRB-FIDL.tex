\documentclass[twocolumn,letterpaper]{scrartcl}

\usepackage[english]{babel}
\usepackage{blindtext}
\usepackage[margin=1in]{geometry}
\usepackage[sorting=none, maxbibnames=99]{biblatex}
\usepackage[breaklinks=true, colorlinks=True, allcolors=blue]{hyperref}
\usepackage{graphicx}
\usepackage[font={small}]{caption}

\addbibresource{CRB.bib}  


% The following lines add color to title, section, and subsection headings.
% Matches color used for FIDLs.
\usepackage{xcolor}
\definecolor{FIDL}{RGB}{31,98,66}
\addtokomafont{title}{\color{FIDL}}
\addtokomafont{section}{\color{FIDL}}
\addtokomafont{subsection}{\color{FIDL}}

\begin{document}

\titlehead{Forest Insect \& Disease Leaflet}	
\title{Coconut Rhinoceros Beetle}
\author{Aubrey Moore}
\maketitle

%\tableofcontents

%\tableofcontents{}
\newpage

\begin{figure}[h]
	\centering
	\includegraphics[width=\linewidth]{images/rhino_beetle_head}
	\caption{Male coconut rhinoceros beetle head and pronotum.}
	\label{fig:rhinobeetlehead}
\end{figure}

Coconut rhinoceros beetle (CRB) (Fig. \ref{fig:rhinobeetlehead}), \textit{Oryctes rhinoceros} (L.) (Coleoptera: Scarabaeidae) is a major pest of coconut palm and oil palm. It is native to southeast Asia, but its has invaded many Pacific islands and its range is currently expanding towards the Americas.

\section{Biology}

\subsection{Taxonomy}

The coconut rhinoceros beetle (CRB), \textit{Oryctes rhinoceros} L., is a member of the scarab beetle  family, Scarabaeidae, and the subfamily Dynastinae. Taxonomic expertise is required to differentiate \textit{O. rhinoceros} from serveral other similar \textit{Oryctes} species, some of which also attack coconut and other palms.


\subsection{Life cycle and reproduction}
CRB has four life stages: eggs, grubs, pupae and adults with the grubs having three substages called instars (Fig. \ref{fig:crblifecycle}). The life span depends on environmental conditions, varying between 9 months and 18 months and generation time varies between 5 months and 9 months.
The CRB sex ratio is usually close to 50:50 and females lay about 65 eggs during their lifetime. Under optimal environmental conditions with an unlimited food supply, CRB populations have the potential to grow at a rate of 3,250\% per generation.

\begin{figure}[h!]
	\centering
	\includegraphics[width=\linewidth]{images/crb_life_cycle}
	\caption{Coconut rhinoceros beetle life cycle.}
	\label{fig:crblifecycle}
\end{figure}

Like all beetles, CRB has four life stages: egg, grub, pupa, and adult (Fig. \ref{fig:crblifecycle}). Only the adult stage causes damage. Grubs feed only on decaying vegetation and do no harm. Adult males and females bore into the crowns of coconut palms and other palms to feed on sap. 

Adults do not feed on leaves, but they bore holes through developing leaves on their way to the white tissue at the interior of the crown. When these damaged leaves eventually emerge from the crown, they have v-shaped cuts in them, a distinctive sign of CRB damage (Fig. \ref{fig:dyingcoconuts}). Each adult feeds on sap for only a few days. It then leaves the crown to search for a breeding site. Palms may be killed if a CRB bores through the growing tip (the meristem). Mature palms are rarely killed at low CRB population levels. However, trees are killed when they are simultaneously attacked by many adults during a population outbreak such as the one we are currently experiencing on Guam. Coconut rhinoceros beetles attain maximum mass at the end of the third instar. Adults are at maximum mass when they have just emerged from the pupa.

All life stages aggregate in CRB breeding sites which can be found wherever vegetation accumulates. Preferred sites are standing dead coconut stems and fallen coconut logs and fronds. But piles of anything with a high concentration of decaying vegetation can be used as a breeding site including green-waste, dead trees of any species, saw dust, and manure. CRB breeding sites have even been found in commercially bagged soil purchased from a local hardware store [3]. 
%An active breeding site will contain all CRB life stages. Adults locate breeding sites by sniffing out a chemical signal referred to as an aggregation pheromone. This pheromone has been synthesized and is commercially available [4].

A female CRB lays about 100 eggs during her lifetime. Assuming a 50\% sex ratio and 100\% survival, there will be a population increase of 5,000\% during each generation. Thus population explosions may occur when abundant potential breeding sites are available in the form of rotting vegetation following destruction in the wake of a typhoon, large scale land clearing, or war. Large numbers of CRB adults generated by a population explosion may result in large numbers of palms being killed. The dead standing trunks soon become ideal breeding sites which generate even higher numbers of adults. This positive feedback loop will end when the rhino beetles run out of food, meaning when most of the palms have been killed and rotted away.

\subsection{Damage caused by CRB}
Grubs feed in almost any type of dead vegetation, including animal manure,  and they do no damage. However, both male and female adults bore into the crowns of palms to feed on sap. Each feeding event lasts only a few days and each adult may feed several times before it dies. Damage is caused when the beetle bores through developing fronds. When these fronds emerge from the crown they exhibit distinctive v-shaped cuts. This damage results in reduced photosynthesis, fruit production and loss of aesthetic quality in ornamental palms. If a bore hole passes through the meristem (growing tip), the palm will not be able to produce new fronds and it will die within about a year when all existing fronds senesce and fall off. Mortality caused by CRB is often seen in young palms but it is rare in mature palms unless there is a high population of CRB adults. 

Some host plant lists for CRB are misleading. Grubs can be found in decaying material from many plant species such as LIST HERE. Accumulation of detritus in coconut palm crowns on Guam. Dead limbs.

Adults will occasionally bore into live plants to feed on sap for a few days. Banana pandanus cycads etc.

\begin{figure}[h]
	\centering
	\includegraphics[width=0.8\linewidth]{images/dying_coconuts}
	\caption{Coconut palms severely attacked by coconut rhinoceros beetle.}
	\label{fig:dyingcoconuts}
\end{figure}

\subsection{Population dynamics}
Dead palms quickly become ideal CRB breeding sites. Large numbers of dead palms killed by CRB adults or  large accumulations of dead vegetation caused by tropical cyclones, massive land clearing or military activity may lead to a self-sustaining  CRB outbreak: CRB adults kill palms which become breeding sites. CRB adults emerging from these breeding sites kill surrounding palms, creating new breeding sites which generate even more adults which kill even more palms. An example of this positive feedback cycle occurred in Palau as a result of massive destruction of palms and other vegetation during the Second World War. Prior to the war, CRB was very rare in Palau but shortly afterwards about 50\% of coconut palms were killed by CRB.

\subsection{Geographic distribution}

CRB invaded islands in the Pacific and Indian Oceans during two waves of movement (Fig. \ref{fig:crbdist}). The first wave occurred started in 1909 when CRB was accidentally transported to from Sri Lanka to Samoa with shipment of rubber tree seedlings and it ended during the 1970s
[5]. All of the CRB range expansion during this period was south of the equator except for the invasion of the Ryuku Islands (Japan) starting in 1921 [6] and invasion of the Palau Islands in about 1942 [5]. In Palau, there was a population explosion of rhino beetles because WWII activities created abundant breeding sites. This resulted in about 50\% coconut palm mortality overall, and total loss of coconut palms on some of the smaller islands [7].

\begin{figure}
	\centering
	\includegraphics[width=\linewidth]{images/crb_dist}
	\caption{Screenshot of an online interactive web map \cite{moore_web_2019} showing the geographic distribution of the coconut rhinoceros beetle. Green markers: native range; Orange markers: first detected during the 20th century; Red markers first detected during the 21st century; Open circle: population includes CRB-G biotype; Filled circle: population is exclusively CRB-G biotype.}
	\label{fig:crbdist}
\end{figure}

The second wave of CRB invasions started in 2007 with discovery of CRB on Guam, followed by invasion of Oahu (Hawaii), Port Morseby (Papua New Guinea), Guadalcanal, Savo and Malaita (Solomon Islands), and Rota (Commonwealth of the Northern Mariana Islands). Beetles in the second wave of invasions are genetically different from those in the first wave [8] and these are being referred to as the Guam biotype or CRB-G for short.

\section{Control actions for CRB}

\subsection{Eradication}

In theory, eradication of CRB from a newly invaded area can be attained by blocking invasion pathways coupled with finding and destroying all breeding sites. In practice, eradication has proven to be very difficult after initial establishment of a CRB population, despite early detection and rapid response. 

Only one of many eradication attempts has succeeded. This was accomplished on the tiny (36 km\textsuperscript{2}) Niuatoputapu Island (also known as Keppel Island), which lies between Samoa and Tonga. During a period spanning 1922 to 1930 all CRB breeding sites were located and destroyed.


%
%CRB is endemic to Southeast Asia where the distinctive v-shaped notches caused by adult feeding are seen frequently in coconut palms but the damage is seldom sufficient to warrant control. The insect becomes a pest, 
%causing severe damage, where there is an abundance of old palm trunks and organic matter, such as that left after cyclones or after felling senile palms for establishment of new plantations (Figure 3.1). In these conditions,female beetles will fly into the area and lay eggs in the organic matter with high numbers of larvae developing in the decaying material. When this generation emerges, the numbers of beetles and the continuing abundance 
%of food resources can lead to a population explosion and emergence of particularly high numbers in the second generation. These numerous emergent adults will cause significant damage to the nearby palms, even killing 
%many, which leads to further substrate availability and an ongoing problem from the beetles. 

\subsection{Sanitation}

Sanitation includes detection and destruction of active and potential CRB breeding sites.

\paragraph{Breeding site detection}

Local searches for breeding sites are usually initiated in response to visible damage to palms or capture of adults in pheromone traps. In Guam and Hawaii, dogs trained to sniff out CRB grubs have been deployed to assist human searchers. Recent research suggests that CRB adults fitted with miniature radio transmitters or harmonic radar tags may be a cost effective way of detecting cryptic breeding sites. The essential idea is that the radio transmitters and tags will accumulate at breeding sites where adults aggregate. They can then be detected by ground and/or aerial surveys using radio receivers or harmonic radar transceivers.


%is a process to remove organic matter sources and is especially important to prevent establishment of new populations and to limit the damage from established populations. CRB larvae feed and grow in rotting organic  matter.  The  life  cycle  can  be  disrupted  and  populations  reduced  if  bulk  sources  of  organic  matter can be removed, or at least reduced. Sanitation methods are reviewed in Godshen (2015) and the sanitation 
%programme carried out on Guam has been described by Smith and Moore (2008). Where sanitation is part of the response to a recent invasion by CRB, it is important to keep thorough records of clean-up activities. In any programme to control CRB, a sanitation plan should be developed and carried out, as outlined below.

\paragraph{Removal of standing dead palms}
CRB adults are attracted to standing dead palm trees that have begun to rot from the crown. Females will lay their eggs in the rotting palm trunk and the developing larvae will feed on the decaying fibers near the top of the trunk, which starts to decompose in the center forming a protective tube for larval development. As the larvae increase in size and strength of their mandibles, they can penetrate further down the trunk leaving a column of frass and cut fibers for the early instars. Dead standing palms should be felled, cut into pieces and burnt or buried to remove potential breeding sites. In some situations, larvae may develop in the crown of live palms. This only occurs where there are large accumulations of organic matter in the frond bases. The organic matter should be removed where possible.

\paragraph{Disposal of dead felled palms}
Mature palm trees will fall after being weakened by fungal diseases (\textit{Ganoderma}), after strong winds during tropical cyclones or after the felling of senile palms prior to replanting. Dead palms on the ground should be cut up into manageable lengths or chipped prior to disposal by burning or deep burial. 

\paragraph{Covering of palm stumps}
Felled palms leave a stump which is suitable for development of larvae as it rots. In management of palm plantations in Asia, where a zero-burning policy is in operation, ground cover is planted shortly after felling to cover the debris and make it less attractive to the flying beetles. The legumes \textit{Mucuna} spp. and \textit{Pueraria javanica} are ground cover plants that are commonly used, as they will add nitrogen to the soil and cover the decaying trunks. 

\paragraph{Management of organic matter and compost}
Heaps  of  organic  matter,  particularly  palm  debris,  provide  excellent  food  material  for  development  of  CRB larvae. Any deep piles of organic material will be attractive to the egg laying females. Heaps of fronds or empty fruit bunches are particularly susceptible. Sawdust from sawmills that process palm timber is also a favorable resource for beetle development. General compost, farmyard manure and even organic garbage can provide sites for development of the larvae. The first step in reducing the threat of beetles emerging from composts is management of the organic matter. Palm debris should be spread among the palms to break down rapidly and release nutrients rather than being piled in heaps. Compost or farmyard manure should be turned regularly, and larvae removed, or pigs and chickens can assist by eating exposed larvae. In urban environments, organic material is often gathered during environmental clean-up and composted, but this may provide a centre for re-infestation of the locality. Compost can be sterilized or fumigated to kill larvae; however, this process is energy-demanding  and  expensive.  Sterile  compost  will  also  be  susceptible  to  re-invasion. Where  feasible,  compost heaps can be covered with netting to trap emerging beetles.

Burning CRB breeding material is the most dependable method for removing the food source for CRB grubs. In Hawaii CRB sanitation programs, breeding site material is being burned on-site using air-curtain burners and some is being trucked to a waste-to-energy electrical power generation plant. 

\subsection{Trapping}

CRB trapping can be used for different purposes including surveillance for early detection, monitoring growth and spread of a population over time, and for population suppression by mass trapping. In all cases, the trap needs to be attractive enough to draw in beetles from a distance strong and enough to contain them once they are captured. Olfactory and visual attractants can be used to increase trap catch. 

\paragraph{Artificial breeding sites} One of the first traps to be developed was the Hoyt trap made from a metal can set on top of a coconut trunk or wooden post. The can was capped with a length of coconut stem with a hole in the center large enough for a beetle to enter. The trap system was used extensively and functioned because it mimicked a standing, decaying coconut stem which is attractive to CRB adults. Another early trap system was simply a pile of coconut log sections placed on the ground.

FIGURE NEEDED: HOYT TRAP, LOG TRAP, PANEL TRAP

\paragraph{Pheromone traps} Design and utility of traps changed with the discovery of ethyl chrysanthemate as an attractant. This was rapidly superseded by ethyl-4-methyloctanoate (E4-MO) commonly referred to as oryctalure, the male-produced aggregation pheromone of CRB, which could be synthesized. E4-MO attracts both sexes, has been used for more than 30 years and is produced commercially by several companies. 

Pheromone traps for surveillance need to be robust, inexpensive, attractive to beetles, difficult to exit and simple to service. Bucket traps, often with vanes, have been used in surveillance trapping in Guam and Hawaii where thousands of traps have been distributed and monitored for delimitation of the spread of CRB populations and to monitor success of control activities. Bucket traps have also been used extensively for monitoring throughout the Pacific Islands and Southeast Asia. Panel traps?

NEED FIGURE OF BUCKET TRAP (VANED AND UNVANED) AND PANEL TRAP

%Construction of a bucket trap
%
%The pheromone bucket trap can be constructed from a plastic bucket with a lid. Two large holes (with a diameter 
%of 2.8 centimetres) and 2 small holes (in the centre) should be made in the lid as illustrated below. The holes can 
%be cut using a hot wire or hot rod. 
%
%The pheromone sachet should be opened and attached with a wire under the bucket lid. The operator should 
%ensure the sachet is not damaged and that it is placed in an upright position inside the bucket. 
%
%The lid should be placed on the bucket (serving as the pheromone trap). In the field, the pheromone trap 
%should be placed on a strong branch with the bucket hanging upright (Prasad S and Lal S, 2006). For diagnostic 
%purposes, such as DNA analysis, beetles should be removed at least once per week and stored individually 
%in plastic containers. If used for monitoring adult beetle numbers, the bucket should be emptied at three-
%week intervals and the collected beetles destroyed. Pheromone sachets need to be replaced every 6–8 weeks 
%(average), given the tropic heat and evaporation levels.
%
%A more costly alternative to the bucket trap is the pipe trap (Figure 3.2D), which was developed by the oil palm 
%industry in PNG and has been widely used. It is made from a two-metre section of PVC piping secured to an 
%iron bar above a collection tray. Two “windows” are cut into the pipe near the top, which aid entry of the beetles 
%before they fall into the collection chamber. Pipe traps are fixed in surveillance sites and suitable for long-term 
%monitoring. 
%
%Sachets of E4-MO can be placed in either bucket or pipe traps as a lure for the beetles. Traps should be placed 
%in a location that is attractive to the flying beetles (on a ridge or in a specific tree) with the aim of obtaining a 
%consistent representative sample of beetles. The trap position should be fixed (by map or GPS) and recorded. 
%The  traps  should  be  cleared  regularly  (usually  every  7–14  days)  with  beetles  differentiated  into  males  and 
%females and numbers recorded. Sachets of lure should be replaced when they appear to have dried out. This 
%depends on weather conditions, but sachets usually last 1–4 months. 

%The E4-MO pheromone is an aggregation pheromone that is attractive to both male and female beetles. The 
%pheromone assists the general orientation of the beetles. However, they are not able to identify the precise 
%location and many will arrive in the vicinity of the trap without entering it. This can result in higher damage to 
%palms close to the trap when compared with the average of the whole plantation.
%
%In the office, trap catches should be analysed, preferably mapped and summarised. The trap catch should be 
%converted to beetles trapped per day, as this figure will be comparable between sites and dates for the same 
%type of trap. An example is presented below, with daily trap catch mapped with time (Figure 3.3). 
%
%Trap information should be complemented with regular photographic assessment of palms around the sample 
%area and satellite images may be used to assist monitoring. Dated satellite images from Google Earth PRO can 
%also be used to provide an indication of changes in damage over time.

\paragraph{Efficacy of pheromone traps for population suppression} Trapping removes adults from the population and can contribute to pest and damage reduction. Bucket traps baited with pheromone have been reported to reduce CRB populations in Malaysia and the related \textit{O. monoceros} in West Africa. 

It has been suggested that mass trapping can eradicated newly detected population of CRB by trapping all adults. Mass trapping was performed on Guam shortly after detection of CRB in the Tumon Bay hotel are in 2007. There was no indication of population suppression and trapping did not reduce damage to palms with the mass trapping areas. During 2010, the trap catch rate in Tumon Bay was only 0.006 beetles per trap day, but CRB damage was visible in 100\% of coconut palms. In contrast, a similar mass trapping program in Samoa trapped 0.150 per trap-day, 25 times the Guam trap-catch rate, but the proportion of damaged coconut palms was only 30\%. Note that the Guam population is the CRB-G biotype and the Samoan population is the CRB-S biotype. 

Three possible explanations have been suggested to account for these observations:
\begin{enumerate}
\item Traps baited with oryctalure are more attractive to CRB-S than CRB-G.
\item CRB-G individuals do far more damage than CRB-S individuals.
\item At very high population levels and trap densities there is so much pheromone in the air that beetles cannot navigate to pheromone sources.
\end{enumerate}

In mark-release-recapture experiments on Guam only 64 of 567 (11\%) of marked beetles where recaptured in a grid of traps baited with oryctalure, indicating that oryctalure is not highly attractive to CRB-G biotype. Unfortunately, there are no comparative data for the CRB-S biotype.  

\paragraph{Tekken fish net traps} Tekken, a gill net used by Chamorro fishermen on Guam, has proven to be an effective trapping tool for coconut rhinoceros beetles. Beetles are captured when a strand of the netting falls into the gap behind a beetle's pronotum, in the same way that fish are caught when a strand falls into a gill slit. In Guam, heaps of organic waste covered with tekken catch beetles emerging from the pile and also attracted to the heaps for mating and oviposition. Cheap and simple pheromone traps can also be made by attaching pieces of tekken to fences and placing an oryctalure dispenser at the center of each piece. 

\begin{figure}[h]
	\centering
	\includegraphics[width=\linewidth]{images/tekken-beetle}
	\includegraphics[width=\linewidth]{images/tekken-pile}
	\includegraphics[width=\linewidth]{images/defence-trap}
	\caption{Tekken.}
	\label{fig:tekken-beetle}
\end{figure}

%Trapping has been used successfully for both monitoring and control. The Hoyt trap was used to monitor the 
%decline of beetle numbers over three years, which was associated with a loss of organic matter for larval feeding 
%(Bedford 1975). Pheromone traps baited with E4-MO can be placed at one trap per 10 hectares to monitor 
%populations in order to establish control action thresholds, or they can be placed at high densities, one per two 
%hectares, in order to reduce CRB populations in plantations (Chung 1997). With improvements in trap design, 
%Moore et al. (2014) estimated the capture of 33 per cent of a CRB population, which would have a significant 
%impact on the population especially if combined with other measures. Tekken netting tied around the palm 
%through  the  leaf  axils  seems  particularly  effective  in  protecting  young  ornamental  palms  from  attack  as  it 
%intercepts the beetles as they move towards the feeding site. 

%\subsection{Mechanical control}
%CRB adults in freshly damaged palms can be located and killed by workers using a wire hook in a process known 
%as winkling. As the beetle bores into the palm petiole (Section 2.5), it produces an excessive amount of cut 
%fibrous material as it drills into the trunk (Fig. 2.8). This will be evident in the axils of the fronds and will indicate 
%the presence of a beetle. A wire can be inserted into the hole and, with skill, the beetle can be removed from 
%the hole and killed. Winkling is mostly used in oil palm where most damage occurs in the crown of young palms 
%and the fresh fibres produced by the beetle can be easily seen. Winkling in coconut palms is more difficult as 
%the worker must climb the palm to see the damage. 

\subsection{Chemical control}

Chemical control of CRB is difficult because all life stages live in protected habitats: all stages are may found inside dead logs or buried in or under heaps of decaying vegetation. Adults may be found boring into palm crowns only briefly, for a few days during feeding bouts. 

\paragraph{Foliar application}
Foliar insecticide application is aimed at preventing damage or mortality of palms caused by adults. The pyrethroid, cypermethrin has been used to successfully protect coconut palms and oil palms. A high enough volume should be applied so that the pesticide runs down the midrib and pools at the base of the petiole where it meets the crownshaft. This is the location at which CRB initiate bore holes. Foliar sprays may be applied by hydraulic sprayers or by pesticide applicator drones.  

\paragraph{Trunk injection}
Trunk injections of systemic insecticides have been applied to oil palms and coconut palms with variable success.

\paragraph{Treatment of breeding sites}
Cypermethrin applied as a drench controls all life stages heaps of loose breeding sites material, but will not kill adults and grubs boring within logs.

\subsection{Biological control}

Biological control is the use of natural enemies (predators, pathogens, parasites) to suppress pest populations. In its native range, CRB is attacked by a community of co-evolved natural enemies, including pathogenic viruses and fungi, predatory carabid and elaterid beetles  and  parasitic  \textit{Scolia}  wasps.  The  relative  impact  of  each  natural  enemy  species  within  this  native community is poorly known, with additional control strategies often needed to complement biological control in coconut and oil palm plantations.

When CRB-S invaded the Pacific, it was the focus of a substantial biological control  program.  The  aim was to find one or more natural enemies in the native range that could be introduced to the invaded range to  suppress  CRB-S  populations.  This  process  is  known  as  classical  biological  control. Among many natural enemies introduced to the Pacific, very few predators or parasites established. Incidental predation by pigs and chickens on CRB larvae may assist with control of this pest and can be useful for control of larvae in household or community waste piles. Local species of generalist arthropod predators (centipedes, beetles, ants) may feed on CRB larvae; however, there is 
little evidence that this contributes significantly to CRB control. 

Only one pathogen provided significant control of CRB-S: the \textit{Oryctes rhinoceros} nudivirus (OrNV) discovered in Malaysia by Alois M. Huger in 1963. This virus infects CRB-S larvae and adults, causing death after 6–30 days. Infected adults are weakened prior to death so that they stop feeding, their mobility is reduced and females stop laying fertile eggs. Once established in countries invaded by CRB-S, the virus significant reduced CRB populations and the damage they caused. 

Another approach to biological control for CRB was to create a biopesticide from a known pathogen. CRB adults and larvae can be infected by the fungus \textit{Metarhizium majus} (formerly \textit{M. anisopliae} var. \textit{majus}). This fungus has been developed into 
a biopesticide that can be applied to CRB breeding sites in both the native and invaded range. 

The recent invasion of CRB-G has changed CRB management wherever it is found. CRB-G is not susceptible to strains of OrNV introduced originally to control CRB-S in the Pacific. A new biological control effort is underway in order to identify strains of OrNV from CRB’s native range that are effective against CRB-G. Until an effective OrNV strain is discovered, biopesticides containing \textit{M. majus} are the only option for biological control of CRB-G. 

\subsection{Integrated pest management}

Integrated pest management (IPM), is a broad-based approach that integrates practices for economic control of  pests.  The  Food and Agriculture Organization of the United Nations   IPM  as  “IPM is the careful consideration of all available pest control techniques and subsequent integration of appropriate measures that discourage the development of pest populations. It combines biological, chemical, physical and crop specific (cultural) management strategies and practices to grow healthy crops and minimize the use of pesticides, reducing or minimizing risks posed by pesticides to human health and the environment for sustainable pest management.”

IPM for management of CRB and includes combination of the factors described above. 

For coconut palms planted for subsistence, ornamental purposes, or in commercial plantations, IPM for CRB involves: 
\begin{enumerate}
\item monitoring of palm damage to detect localized CRB outbreaks and to check that control is successful
\item biological control with OrNV (in the invaded and native range) and other co-evolved natural enemies (in the native range only)
\item sanitation to remove organic waste, dead palms and  other  potential  breeding  sites
\end{enumerate}

Sanitation  is  an  essential  component  of  IPM  for  CRB  that  complements biological control. Localized CRB outbreaks will occur when breeding sites are left uncontrolled, especially after cyclones and tropical storms, when palms are often toppled by high winds and large amounts of green waste is created. Historically, outbreaks of CRB often follow cyclone damage or large scale land clearing. Larvae will develop in the decaying fronds, the trunk and even the roots of the felled palms and also in many other forms of decaying vegetation. 

Some additional options may be incorporated into IPM programmes for coconut, particularly for commercial plantations or ornamental palms. These more costly options include pheromone traps to monitor adult beetle activity and complement it with visual surveys of palm damage. Occasionally, trap catches may be high enough to  contribute  to  population  suppression  in  coconut  plantations,  but  this  strategy  is  more relevant to oil palm (discussed below). Commercial products containing the fungus \textit{M. majus} may be applied to CRB breeding sites that cannot be removed. Insecticide treatments are not recommended for established coconut palms as there is a potential risk of translocation of insecticides (move within the palm), leaving harmful residues in the coconut. If a recent invasion of CRB is targeted for eradication, insecticides may  be  necessary  for  success.  In  this  situation,  expert  advice  is  needed  to  determine  the  most 
appropriate choice of insecticide, to advise on the length of time residues will persist, and to ensure insecticide-contaminated coconuts are not harvested for human consumption. 

For higher value crops, particularly oil palm, the same components are needed as for coconut: monitoring, biological control; and sanitation. More costly IPM components are recommended for oil palm because the crop’s financial value makes greater investment in control worthwhile. Thus, IPM for CRB in oil palm involves: 
\begin{enumerate}
\item monitoring of beetle activity with pheromone traps (section 3.2) and palm damage particularly for young palms
\item biological control with OrNV (invaded and native range) and other natural enemies (native range) plus application of \textit{M. majus} biopesticides to breeding sites 
that cannot be removed
\item sanitation to remove organic waste, particularly during plantation 
renewal when large amounts of waste is generated
\item insecticide treatmentsfor young palms that 
are most sensitive to CRB damage.
\end{enumerate} 

Note that pollinators of oil palm are vulnerable to insecticides, so applications 
should be scheduled carefully to avoid flowering. When oil palm plantations are renewed, complete clean-up 
of the organic waste is challenging. In CRB’s native range, an additional strategy is to break up and spread the 
waste as a thin layer, then plant a fast-growing cover crop, often a legume, over the waste matter (Wood 1968). 

A decision tree to identify IPM options for coconut and oil palms is presented below (Figure 3.5). This includes 
decision  points  to  consider  potential  for  eradication  of  recent  CRB  invasions  as  well  as  management  of 
established CRB populations. 

Figure 3.5   IPM decision tree for CRB control in its invasive range. This tree can be used for either CRB-S  
or CRB-G; however, note that an effective strain of Oryctes virus has not yet been identified for CRB-G. 

\newpage
\section*{References}

\paragraph{Bedford, Geoffrey O. 1980.} Biology, ecology, and control of palm rhinoceros beetles. Annual Review of Entomology 25 (1980): 309–39.
Available online at \url{https://tinyurl.com/yh79wmwc}

\paragraph{Gressitt, J. Linsley 1953.} The coconut rhinoceros beetle (\textit{Oryctes rhinoceros}) with particular reference to the Palau Islands. Bernice P. Bishop Museum. Bulletin 212. 
Available online at \url{https://tinyurl.com/npsab5d4}

\paragraph{Jackson, Trevor, Sean Marshall, Sarah Mansfield and Fereti Atumurirava 2020.} Coconut rhinoceros beetle (\textit{Oryctes rhinoceros}): A manual for control and management of the pest in Pacific Island countries and territories. Pacific Community (SPC). 
Available online at \url{https://tinyurl.com/yxk4u27j} 

\paragraph{Marshall, Sean D. G., Aubrey Moore, Maclean Vaqalo, Alasdair Noble, and Trevor A. Jackson 2017.} A new haplotype of the coconut rhinoceros beetle, \textit{Oryctes rhinoceros}, has escaped biological control by \textit{Oryctes rhinoceros} nudivirus and is invading Pacific Islands. Journal of Invertebrate Pathology, 149, 127-134.
Available online at \url{https://tinyurl.com/mtpp29da}

\paragraph{Pallipparambil, Godshen R. 2015.} New Pest Response Guidelines: \textit{Oryctes rhinoceros} (L.) Coleoptera: Scarabaeidae, Coconut rhinoceros beetle. U.S. Department of Agriculture, Animal Plant Health Inspection Service, Plant Protection and Quarantine.
Available online at \url{https://tinyurl.com/mpdmvfpt}

%\nocite{jackson_coconut_2020}
%
%
%\printbibliography

\end{document}
